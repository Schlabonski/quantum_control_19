
% Default to the notebook output style

    


% Inherit from the specified cell style.




    
\documentclass[11pt]{article}

    
    
    \usepackage[T1]{fontenc}
    % Nicer default font (+ math font) than Computer Modern for most use cases
    \usepackage{mathpazo}

    % Basic figure setup, for now with no caption control since it's done
    % automatically by Pandoc (which extracts ![](path) syntax from Markdown).
    \usepackage{graphicx}
    % We will generate all images so they have a width \maxwidth. This means
    % that they will get their normal width if they fit onto the page, but
    % are scaled down if they would overflow the margins.
    \makeatletter
    \def\maxwidth{\ifdim\Gin@nat@width>\linewidth\linewidth
    \else\Gin@nat@width\fi}
    \makeatother
    \let\Oldincludegraphics\includegraphics
    % Set max figure width to be 80% of text width, for now hardcoded.
    \renewcommand{\includegraphics}[1]{\Oldincludegraphics[width=.8\maxwidth]{#1}}
    % Ensure that by default, figures have no caption (until we provide a
    % proper Figure object with a Caption API and a way to capture that
    % in the conversion process - todo).
    \usepackage{caption}
    \DeclareCaptionLabelFormat{nolabel}{}
    \captionsetup{labelformat=nolabel}

    \usepackage{adjustbox} % Used to constrain images to a maximum size 
    \usepackage{xcolor} % Allow colors to be defined
    \usepackage{enumerate} % Needed for markdown enumerations to work
    \usepackage{geometry} % Used to adjust the document margins
    \usepackage{amsmath} % Equations
    \usepackage{amssymb} % Equations
    \usepackage{textcomp} % defines textquotesingle
    % Hack from http://tex.stackexchange.com/a/47451/13684:
    \AtBeginDocument{%
        \def\PYZsq{\textquotesingle}% Upright quotes in Pygmentized code
    }
    \usepackage{upquote} % Upright quotes for verbatim code
    \usepackage{eurosym} % defines \euro
    \usepackage[mathletters]{ucs} % Extended unicode (utf-8) support
    \usepackage[utf8x]{inputenc} % Allow utf-8 characters in the tex document
    \usepackage{fancyvrb} % verbatim replacement that allows latex
    \usepackage{grffile} % extends the file name processing of package graphics 
                         % to support a larger range 
    % The hyperref package gives us a pdf with properly built
    % internal navigation ('pdf bookmarks' for the table of contents,
    % internal cross-reference links, web links for URLs, etc.)
    \usepackage{hyperref}
    \usepackage{longtable} % longtable support required by pandoc >1.10
    \usepackage{booktabs}  % table support for pandoc > 1.12.2
    \usepackage[inline]{enumitem} % IRkernel/repr support (it uses the enumerate* environment)
    \usepackage[normalem]{ulem} % ulem is needed to support strikethroughs (\sout)
                                % normalem makes italics be italics, not underlines
    

    
    
    % Colors for the hyperref package
    \definecolor{urlcolor}{rgb}{0,.145,.698}
    \definecolor{linkcolor}{rgb}{.71,0.21,0.01}
    \definecolor{citecolor}{rgb}{.12,.54,.11}

    % ANSI colors
    \definecolor{ansi-black}{HTML}{3E424D}
    \definecolor{ansi-black-intense}{HTML}{282C36}
    \definecolor{ansi-red}{HTML}{E75C58}
    \definecolor{ansi-red-intense}{HTML}{B22B31}
    \definecolor{ansi-green}{HTML}{00A250}
    \definecolor{ansi-green-intense}{HTML}{007427}
    \definecolor{ansi-yellow}{HTML}{DDB62B}
    \definecolor{ansi-yellow-intense}{HTML}{B27D12}
    \definecolor{ansi-blue}{HTML}{208FFB}
    \definecolor{ansi-blue-intense}{HTML}{0065CA}
    \definecolor{ansi-magenta}{HTML}{D160C4}
    \definecolor{ansi-magenta-intense}{HTML}{A03196}
    \definecolor{ansi-cyan}{HTML}{60C6C8}
    \definecolor{ansi-cyan-intense}{HTML}{258F8F}
    \definecolor{ansi-white}{HTML}{C5C1B4}
    \definecolor{ansi-white-intense}{HTML}{A1A6B2}

    % commands and environments needed by pandoc snippets
    % extracted from the output of `pandoc -s`
    \providecommand{\tightlist}{%
      \setlength{\itemsep}{0pt}\setlength{\parskip}{0pt}}
    \DefineVerbatimEnvironment{Highlighting}{Verbatim}{commandchars=\\\{\}}
    % Add ',fontsize=\small' for more characters per line
    \newenvironment{Shaded}{}{}
    \newcommand{\KeywordTok}[1]{\textcolor[rgb]{0.00,0.44,0.13}{\textbf{{#1}}}}
    \newcommand{\DataTypeTok}[1]{\textcolor[rgb]{0.56,0.13,0.00}{{#1}}}
    \newcommand{\DecValTok}[1]{\textcolor[rgb]{0.25,0.63,0.44}{{#1}}}
    \newcommand{\BaseNTok}[1]{\textcolor[rgb]{0.25,0.63,0.44}{{#1}}}
    \newcommand{\FloatTok}[1]{\textcolor[rgb]{0.25,0.63,0.44}{{#1}}}
    \newcommand{\CharTok}[1]{\textcolor[rgb]{0.25,0.44,0.63}{{#1}}}
    \newcommand{\StringTok}[1]{\textcolor[rgb]{0.25,0.44,0.63}{{#1}}}
    \newcommand{\CommentTok}[1]{\textcolor[rgb]{0.38,0.63,0.69}{\textit{{#1}}}}
    \newcommand{\OtherTok}[1]{\textcolor[rgb]{0.00,0.44,0.13}{{#1}}}
    \newcommand{\AlertTok}[1]{\textcolor[rgb]{1.00,0.00,0.00}{\textbf{{#1}}}}
    \newcommand{\FunctionTok}[1]{\textcolor[rgb]{0.02,0.16,0.49}{{#1}}}
    \newcommand{\RegionMarkerTok}[1]{{#1}}
    \newcommand{\ErrorTok}[1]{\textcolor[rgb]{1.00,0.00,0.00}{\textbf{{#1}}}}
    \newcommand{\NormalTok}[1]{{#1}}
    
    % Additional commands for more recent versions of Pandoc
    \newcommand{\ConstantTok}[1]{\textcolor[rgb]{0.53,0.00,0.00}{{#1}}}
    \newcommand{\SpecialCharTok}[1]{\textcolor[rgb]{0.25,0.44,0.63}{{#1}}}
    \newcommand{\VerbatimStringTok}[1]{\textcolor[rgb]{0.25,0.44,0.63}{{#1}}}
    \newcommand{\SpecialStringTok}[1]{\textcolor[rgb]{0.73,0.40,0.53}{{#1}}}
    \newcommand{\ImportTok}[1]{{#1}}
    \newcommand{\DocumentationTok}[1]{\textcolor[rgb]{0.73,0.13,0.13}{\textit{{#1}}}}
    \newcommand{\AnnotationTok}[1]{\textcolor[rgb]{0.38,0.63,0.69}{\textbf{\textit{{#1}}}}}
    \newcommand{\CommentVarTok}[1]{\textcolor[rgb]{0.38,0.63,0.69}{\textbf{\textit{{#1}}}}}
    \newcommand{\VariableTok}[1]{\textcolor[rgb]{0.10,0.09,0.49}{{#1}}}
    \newcommand{\ControlFlowTok}[1]{\textcolor[rgb]{0.00,0.44,0.13}{\textbf{{#1}}}}
    \newcommand{\OperatorTok}[1]{\textcolor[rgb]{0.40,0.40,0.40}{{#1}}}
    \newcommand{\BuiltInTok}[1]{{#1}}
    \newcommand{\ExtensionTok}[1]{{#1}}
    \newcommand{\PreprocessorTok}[1]{\textcolor[rgb]{0.74,0.48,0.00}{{#1}}}
    \newcommand{\AttributeTok}[1]{\textcolor[rgb]{0.49,0.56,0.16}{{#1}}}
    \newcommand{\InformationTok}[1]{\textcolor[rgb]{0.38,0.63,0.69}{\textbf{\textit{{#1}}}}}
    \newcommand{\WarningTok}[1]{\textcolor[rgb]{0.38,0.63,0.69}{\textbf{\textit{{#1}}}}}
    
    
    % Define a nice break command that doesn't care if a line doesn't already
    % exist.
    \def\br{\hspace*{\fill} \\* }
    % Math Jax compatability definitions
    \def\gt{>}
    \def\lt{<}
    % Document parameters
    \title{sheet12\_quantum\_algorithms\_solution}
    
    
    

    % Pygments definitions
    
\makeatletter
\def\PY@reset{\let\PY@it=\relax \let\PY@bf=\relax%
    \let\PY@ul=\relax \let\PY@tc=\relax%
    \let\PY@bc=\relax \let\PY@ff=\relax}
\def\PY@tok#1{\csname PY@tok@#1\endcsname}
\def\PY@toks#1+{\ifx\relax#1\empty\else%
    \PY@tok{#1}\expandafter\PY@toks\fi}
\def\PY@do#1{\PY@bc{\PY@tc{\PY@ul{%
    \PY@it{\PY@bf{\PY@ff{#1}}}}}}}
\def\PY#1#2{\PY@reset\PY@toks#1+\relax+\PY@do{#2}}

\expandafter\def\csname PY@tok@w\endcsname{\def\PY@tc##1{\textcolor[rgb]{0.73,0.73,0.73}{##1}}}
\expandafter\def\csname PY@tok@c\endcsname{\let\PY@it=\textit\def\PY@tc##1{\textcolor[rgb]{0.25,0.50,0.50}{##1}}}
\expandafter\def\csname PY@tok@cp\endcsname{\def\PY@tc##1{\textcolor[rgb]{0.74,0.48,0.00}{##1}}}
\expandafter\def\csname PY@tok@k\endcsname{\let\PY@bf=\textbf\def\PY@tc##1{\textcolor[rgb]{0.00,0.50,0.00}{##1}}}
\expandafter\def\csname PY@tok@kp\endcsname{\def\PY@tc##1{\textcolor[rgb]{0.00,0.50,0.00}{##1}}}
\expandafter\def\csname PY@tok@kt\endcsname{\def\PY@tc##1{\textcolor[rgb]{0.69,0.00,0.25}{##1}}}
\expandafter\def\csname PY@tok@o\endcsname{\def\PY@tc##1{\textcolor[rgb]{0.40,0.40,0.40}{##1}}}
\expandafter\def\csname PY@tok@ow\endcsname{\let\PY@bf=\textbf\def\PY@tc##1{\textcolor[rgb]{0.67,0.13,1.00}{##1}}}
\expandafter\def\csname PY@tok@nb\endcsname{\def\PY@tc##1{\textcolor[rgb]{0.00,0.50,0.00}{##1}}}
\expandafter\def\csname PY@tok@nf\endcsname{\def\PY@tc##1{\textcolor[rgb]{0.00,0.00,1.00}{##1}}}
\expandafter\def\csname PY@tok@nc\endcsname{\let\PY@bf=\textbf\def\PY@tc##1{\textcolor[rgb]{0.00,0.00,1.00}{##1}}}
\expandafter\def\csname PY@tok@nn\endcsname{\let\PY@bf=\textbf\def\PY@tc##1{\textcolor[rgb]{0.00,0.00,1.00}{##1}}}
\expandafter\def\csname PY@tok@ne\endcsname{\let\PY@bf=\textbf\def\PY@tc##1{\textcolor[rgb]{0.82,0.25,0.23}{##1}}}
\expandafter\def\csname PY@tok@nv\endcsname{\def\PY@tc##1{\textcolor[rgb]{0.10,0.09,0.49}{##1}}}
\expandafter\def\csname PY@tok@no\endcsname{\def\PY@tc##1{\textcolor[rgb]{0.53,0.00,0.00}{##1}}}
\expandafter\def\csname PY@tok@nl\endcsname{\def\PY@tc##1{\textcolor[rgb]{0.63,0.63,0.00}{##1}}}
\expandafter\def\csname PY@tok@ni\endcsname{\let\PY@bf=\textbf\def\PY@tc##1{\textcolor[rgb]{0.60,0.60,0.60}{##1}}}
\expandafter\def\csname PY@tok@na\endcsname{\def\PY@tc##1{\textcolor[rgb]{0.49,0.56,0.16}{##1}}}
\expandafter\def\csname PY@tok@nt\endcsname{\let\PY@bf=\textbf\def\PY@tc##1{\textcolor[rgb]{0.00,0.50,0.00}{##1}}}
\expandafter\def\csname PY@tok@nd\endcsname{\def\PY@tc##1{\textcolor[rgb]{0.67,0.13,1.00}{##1}}}
\expandafter\def\csname PY@tok@s\endcsname{\def\PY@tc##1{\textcolor[rgb]{0.73,0.13,0.13}{##1}}}
\expandafter\def\csname PY@tok@sd\endcsname{\let\PY@it=\textit\def\PY@tc##1{\textcolor[rgb]{0.73,0.13,0.13}{##1}}}
\expandafter\def\csname PY@tok@si\endcsname{\let\PY@bf=\textbf\def\PY@tc##1{\textcolor[rgb]{0.73,0.40,0.53}{##1}}}
\expandafter\def\csname PY@tok@se\endcsname{\let\PY@bf=\textbf\def\PY@tc##1{\textcolor[rgb]{0.73,0.40,0.13}{##1}}}
\expandafter\def\csname PY@tok@sr\endcsname{\def\PY@tc##1{\textcolor[rgb]{0.73,0.40,0.53}{##1}}}
\expandafter\def\csname PY@tok@ss\endcsname{\def\PY@tc##1{\textcolor[rgb]{0.10,0.09,0.49}{##1}}}
\expandafter\def\csname PY@tok@sx\endcsname{\def\PY@tc##1{\textcolor[rgb]{0.00,0.50,0.00}{##1}}}
\expandafter\def\csname PY@tok@m\endcsname{\def\PY@tc##1{\textcolor[rgb]{0.40,0.40,0.40}{##1}}}
\expandafter\def\csname PY@tok@gh\endcsname{\let\PY@bf=\textbf\def\PY@tc##1{\textcolor[rgb]{0.00,0.00,0.50}{##1}}}
\expandafter\def\csname PY@tok@gu\endcsname{\let\PY@bf=\textbf\def\PY@tc##1{\textcolor[rgb]{0.50,0.00,0.50}{##1}}}
\expandafter\def\csname PY@tok@gd\endcsname{\def\PY@tc##1{\textcolor[rgb]{0.63,0.00,0.00}{##1}}}
\expandafter\def\csname PY@tok@gi\endcsname{\def\PY@tc##1{\textcolor[rgb]{0.00,0.63,0.00}{##1}}}
\expandafter\def\csname PY@tok@gr\endcsname{\def\PY@tc##1{\textcolor[rgb]{1.00,0.00,0.00}{##1}}}
\expandafter\def\csname PY@tok@ge\endcsname{\let\PY@it=\textit}
\expandafter\def\csname PY@tok@gs\endcsname{\let\PY@bf=\textbf}
\expandafter\def\csname PY@tok@gp\endcsname{\let\PY@bf=\textbf\def\PY@tc##1{\textcolor[rgb]{0.00,0.00,0.50}{##1}}}
\expandafter\def\csname PY@tok@go\endcsname{\def\PY@tc##1{\textcolor[rgb]{0.53,0.53,0.53}{##1}}}
\expandafter\def\csname PY@tok@gt\endcsname{\def\PY@tc##1{\textcolor[rgb]{0.00,0.27,0.87}{##1}}}
\expandafter\def\csname PY@tok@err\endcsname{\def\PY@bc##1{\setlength{\fboxsep}{0pt}\fcolorbox[rgb]{1.00,0.00,0.00}{1,1,1}{\strut ##1}}}
\expandafter\def\csname PY@tok@kc\endcsname{\let\PY@bf=\textbf\def\PY@tc##1{\textcolor[rgb]{0.00,0.50,0.00}{##1}}}
\expandafter\def\csname PY@tok@kd\endcsname{\let\PY@bf=\textbf\def\PY@tc##1{\textcolor[rgb]{0.00,0.50,0.00}{##1}}}
\expandafter\def\csname PY@tok@kn\endcsname{\let\PY@bf=\textbf\def\PY@tc##1{\textcolor[rgb]{0.00,0.50,0.00}{##1}}}
\expandafter\def\csname PY@tok@kr\endcsname{\let\PY@bf=\textbf\def\PY@tc##1{\textcolor[rgb]{0.00,0.50,0.00}{##1}}}
\expandafter\def\csname PY@tok@bp\endcsname{\def\PY@tc##1{\textcolor[rgb]{0.00,0.50,0.00}{##1}}}
\expandafter\def\csname PY@tok@fm\endcsname{\def\PY@tc##1{\textcolor[rgb]{0.00,0.00,1.00}{##1}}}
\expandafter\def\csname PY@tok@vc\endcsname{\def\PY@tc##1{\textcolor[rgb]{0.10,0.09,0.49}{##1}}}
\expandafter\def\csname PY@tok@vg\endcsname{\def\PY@tc##1{\textcolor[rgb]{0.10,0.09,0.49}{##1}}}
\expandafter\def\csname PY@tok@vi\endcsname{\def\PY@tc##1{\textcolor[rgb]{0.10,0.09,0.49}{##1}}}
\expandafter\def\csname PY@tok@vm\endcsname{\def\PY@tc##1{\textcolor[rgb]{0.10,0.09,0.49}{##1}}}
\expandafter\def\csname PY@tok@sa\endcsname{\def\PY@tc##1{\textcolor[rgb]{0.73,0.13,0.13}{##1}}}
\expandafter\def\csname PY@tok@sb\endcsname{\def\PY@tc##1{\textcolor[rgb]{0.73,0.13,0.13}{##1}}}
\expandafter\def\csname PY@tok@sc\endcsname{\def\PY@tc##1{\textcolor[rgb]{0.73,0.13,0.13}{##1}}}
\expandafter\def\csname PY@tok@dl\endcsname{\def\PY@tc##1{\textcolor[rgb]{0.73,0.13,0.13}{##1}}}
\expandafter\def\csname PY@tok@s2\endcsname{\def\PY@tc##1{\textcolor[rgb]{0.73,0.13,0.13}{##1}}}
\expandafter\def\csname PY@tok@sh\endcsname{\def\PY@tc##1{\textcolor[rgb]{0.73,0.13,0.13}{##1}}}
\expandafter\def\csname PY@tok@s1\endcsname{\def\PY@tc##1{\textcolor[rgb]{0.73,0.13,0.13}{##1}}}
\expandafter\def\csname PY@tok@mb\endcsname{\def\PY@tc##1{\textcolor[rgb]{0.40,0.40,0.40}{##1}}}
\expandafter\def\csname PY@tok@mf\endcsname{\def\PY@tc##1{\textcolor[rgb]{0.40,0.40,0.40}{##1}}}
\expandafter\def\csname PY@tok@mh\endcsname{\def\PY@tc##1{\textcolor[rgb]{0.40,0.40,0.40}{##1}}}
\expandafter\def\csname PY@tok@mi\endcsname{\def\PY@tc##1{\textcolor[rgb]{0.40,0.40,0.40}{##1}}}
\expandafter\def\csname PY@tok@il\endcsname{\def\PY@tc##1{\textcolor[rgb]{0.40,0.40,0.40}{##1}}}
\expandafter\def\csname PY@tok@mo\endcsname{\def\PY@tc##1{\textcolor[rgb]{0.40,0.40,0.40}{##1}}}
\expandafter\def\csname PY@tok@ch\endcsname{\let\PY@it=\textit\def\PY@tc##1{\textcolor[rgb]{0.25,0.50,0.50}{##1}}}
\expandafter\def\csname PY@tok@cm\endcsname{\let\PY@it=\textit\def\PY@tc##1{\textcolor[rgb]{0.25,0.50,0.50}{##1}}}
\expandafter\def\csname PY@tok@cpf\endcsname{\let\PY@it=\textit\def\PY@tc##1{\textcolor[rgb]{0.25,0.50,0.50}{##1}}}
\expandafter\def\csname PY@tok@c1\endcsname{\let\PY@it=\textit\def\PY@tc##1{\textcolor[rgb]{0.25,0.50,0.50}{##1}}}
\expandafter\def\csname PY@tok@cs\endcsname{\let\PY@it=\textit\def\PY@tc##1{\textcolor[rgb]{0.25,0.50,0.50}{##1}}}

\def\PYZbs{\char`\\}
\def\PYZus{\char`\_}
\def\PYZob{\char`\{}
\def\PYZcb{\char`\}}
\def\PYZca{\char`\^}
\def\PYZam{\char`\&}
\def\PYZlt{\char`\<}
\def\PYZgt{\char`\>}
\def\PYZsh{\char`\#}
\def\PYZpc{\char`\%}
\def\PYZdl{\char`\$}
\def\PYZhy{\char`\-}
\def\PYZsq{\char`\'}
\def\PYZdq{\char`\"}
\def\PYZti{\char`\~}
% for compatibility with earlier versions
\def\PYZat{@}
\def\PYZlb{[}
\def\PYZrb{]}
\makeatother


    % Exact colors from NB
    \definecolor{incolor}{rgb}{0.0, 0.0, 0.5}
    \definecolor{outcolor}{rgb}{0.545, 0.0, 0.0}



    
    % Prevent overflowing lines due to hard-to-break entities
    \sloppy 
    % Setup hyperref package
    \hypersetup{
      breaklinks=true,  % so long urls are correctly broken across lines
      colorlinks=true,
      urlcolor=urlcolor,
      linkcolor=linkcolor,
      citecolor=citecolor,
      }
    % Slightly bigger margins than the latex defaults
    
    \geometry{verbose,tmargin=1in,bmargin=1in,lmargin=1in,rmargin=1in}
    
    

    \begin{document}
    
    
    \maketitle
    
    

    
    \section{Sheet 12 - Numerical Part}\label{sheet-12---numerical-part}

To be handed in until July 17th, 18:00. Please send your solutions to
pascal.weckesser@physik.uni-freiburg.de.

    \begin{Verbatim}[commandchars=\\\{\}]
{\color{incolor}In [{\color{incolor}409}]:} \PY{o}{\PYZpc{}}\PY{k}{matplotlib} notebook
          \PY{k+kn}{import} \PY{n+nn}{numpy} \PY{k}{as} \PY{n+nn}{np}
          \PY{k+kn}{import} \PY{n+nn}{matplotlib}\PY{n+nn}{.}\PY{n+nn}{pyplot} \PY{k}{as} \PY{n+nn}{plt}
          
          \PY{k+kn}{import} \PY{n+nn}{qutip} \PY{k}{as} \PY{n+nn}{q}
          \PY{n}{pi} \PY{o}{=} \PY{n}{np}\PY{o}{.}\PY{n}{pi}
          
          \PY{k+kn}{from} \PY{n+nn}{IPython}\PY{n+nn}{.}\PY{n+nn}{core}\PY{n+nn}{.}\PY{n+nn}{display} \PY{k}{import} \PY{n}{SVG} 
\end{Verbatim}


    \[\newcommand{\bra}[1]{\left\langle\,{#1}\,\right|}
\newcommand{\ket}[1]{\left|\,{#1}\,\right\rangle}
\newcommand{\braket}[2]{\left\langle{#1}\middle|{#2}\right\rangle}\] \#
Spin Flip Correction

    \texttt{qutip} has the ability to compute the effect of multi-qubit
gates of general input states and give a latex representation of the
algorithm. This can look like this:

    \begin{Verbatim}[commandchars=\\\{\}]
{\color{incolor}In [{\color{incolor}410}]:} \PY{n}{qc} \PY{o}{=} \PY{n}{q}\PY{o}{.}\PY{n}{QubitCircuit}\PY{p}{(}\PY{n}{N}\PY{o}{=}\PY{l+m+mi}{2}\PY{p}{)}
          
          \PY{c+c1}{\PYZsh{} input states defined here are only used for plotting!}
          \PY{n}{qc}\PY{o}{.}\PY{n}{add\PYZus{}state}\PY{p}{(}\PY{l+s+s1}{\PYZsq{}}\PY{l+s+s1}{0}\PY{l+s+s1}{\PYZsq{}}\PY{p}{,} \PY{p}{[}\PY{l+m+mi}{0}\PY{p}{]}\PY{p}{)}
          \PY{n}{qc}\PY{o}{.}\PY{n}{add\PYZus{}state}\PY{p}{(}\PY{l+s+s1}{\PYZsq{}}\PY{l+s+s1}{1}\PY{l+s+s1}{\PYZsq{}}\PY{p}{,} \PY{p}{[}\PY{l+m+mi}{1}\PY{p}{]}\PY{p}{)}
          
          \PY{c+c1}{\PYZsh{} add a Hadamard / SNOT gate for qubit 0. Pay attention, the counting starts from the}
          \PY{c+c1}{\PYZsh{} most significant (= first in terms of binary represenation) bit.}
          \PY{n}{qc}\PY{o}{.}\PY{n}{add\PYZus{}gate}\PY{p}{(}\PY{l+s+s1}{\PYZsq{}}\PY{l+s+s1}{SNOT}\PY{l+s+s1}{\PYZsq{}}\PY{p}{,} \PY{n}{controls}\PY{o}{=}\PY{k+kc}{None}\PY{p}{,} \PY{n}{targets}\PY{o}{=}\PY{l+m+mi}{0}\PY{p}{)}
          
          \PY{c+c1}{\PYZsh{} entangle the two qubits}
          \PY{n}{qc}\PY{o}{.}\PY{n}{add\PYZus{}gate}\PY{p}{(}\PY{l+s+s1}{\PYZsq{}}\PY{l+s+s1}{CNOT}\PY{l+s+s1}{\PYZsq{}}\PY{p}{,} \PY{n}{controls}\PY{o}{=}\PY{l+m+mi}{0}\PY{p}{,} \PY{n}{targets}\PY{o}{=}\PY{l+m+mi}{1}\PY{p}{)}
          
          \PY{c+c1}{\PYZsh{} this allows us to create a neat representation of the algorithm}
          \PY{n}{qc}\PY{o}{.}\PY{n}{png}
\end{Verbatim}

\texttt{\color{outcolor}Out[{\color{outcolor}410}]:}
    
    \begin{center}
    \adjustimage{max size={0.9\linewidth}{0.9\paperheight}}{output_4_0.png}
    \end{center}
    { \hspace*{\fill} \\}
    

    \emph{Remark:} \texttt{qc.png} is possibly not working due to
\href{https://bugs.launchpad.net/ubuntu/+source/imagemagick/+bug/1796563}{this
bug}. However a \texttt{.pdf} should be created if you have latex
installed.

    The effect of the gate on a state can be computed as in the following
example:

    \begin{Verbatim}[commandchars=\\\{\}]
{\color{incolor}In [{\color{incolor}411}]:} \PY{c+c1}{\PYZsh{} sorry for the redundance here}
          \PY{k}{def} \PY{n+nf}{five\PYZus{}bit\PYZus{}basis\PYZus{}representation}\PY{p}{(}\PY{n}{ket}\PY{p}{)}\PY{p}{:}
              \PY{k}{for} \PY{n}{i}\PY{p}{,} \PY{n}{s} \PY{o+ow}{in} \PY{n+nb}{enumerate}\PY{p}{(}\PY{n}{ket}\PY{p}{)}\PY{p}{:}
                  \PY{n+nb}{print}\PY{p}{(}\PY{l+s+s1}{\PYZsq{}}\PY{l+s+s1}{|}\PY{l+s+si}{\PYZob{}0:05b\PYZcb{}}\PY{l+s+s1}{\PYZgt{} : }\PY{l+s+si}{\PYZob{}1:.3f\PYZcb{}}\PY{l+s+s1}{\PYZsq{}}\PY{o}{.}\PY{n}{format}\PY{p}{(}\PY{n}{i}\PY{p}{,} \PY{n}{s}\PY{p}{[}\PY{l+m+mi}{0}\PY{p}{,}\PY{l+m+mi}{0}\PY{p}{]}\PY{p}{)}\PY{p}{)}
                  
          \PY{k}{def} \PY{n+nf}{three\PYZus{}bit\PYZus{}basis\PYZus{}representation}\PY{p}{(}\PY{n}{ket}\PY{p}{)}\PY{p}{:}
              \PY{k}{for} \PY{n}{i}\PY{p}{,} \PY{n}{s} \PY{o+ow}{in} \PY{n+nb}{enumerate}\PY{p}{(}\PY{n}{ket}\PY{p}{)}\PY{p}{:}
                  \PY{n+nb}{print}\PY{p}{(}\PY{l+s+s1}{\PYZsq{}}\PY{l+s+s1}{|}\PY{l+s+si}{\PYZob{}0:03b\PYZcb{}}\PY{l+s+s1}{\PYZgt{} : }\PY{l+s+si}{\PYZob{}1:.3f\PYZcb{}}\PY{l+s+s1}{\PYZsq{}}\PY{o}{.}\PY{n}{format}\PY{p}{(}\PY{n}{i}\PY{p}{,} \PY{n}{s}\PY{p}{[}\PY{l+m+mi}{0}\PY{p}{,}\PY{l+m+mi}{0}\PY{p}{]}\PY{p}{)}\PY{p}{)}
                  
          \PY{k}{def} \PY{n+nf}{two\PYZus{}bit\PYZus{}basis\PYZus{}representation}\PY{p}{(}\PY{n}{ket}\PY{p}{)}\PY{p}{:}
              \PY{k}{for} \PY{n}{i}\PY{p}{,} \PY{n}{s} \PY{o+ow}{in} \PY{n+nb}{enumerate}\PY{p}{(}\PY{n}{ket}\PY{p}{)}\PY{p}{:}
                  \PY{n+nb}{print}\PY{p}{(}\PY{l+s+s1}{\PYZsq{}}\PY{l+s+s1}{|}\PY{l+s+si}{\PYZob{}0:02b\PYZcb{}}\PY{l+s+s1}{\PYZgt{} : }\PY{l+s+si}{\PYZob{}1:.3f\PYZcb{}}\PY{l+s+s1}{\PYZsq{}}\PY{o}{.}\PY{n}{format}\PY{p}{(}\PY{n}{i}\PY{p}{,} \PY{n}{s}\PY{p}{[}\PY{l+m+mi}{0}\PY{p}{,}\PY{l+m+mi}{0}\PY{p}{]}\PY{p}{)}\PY{p}{)}
\end{Verbatim}


    \begin{Verbatim}[commandchars=\\\{\}]
{\color{incolor}In [{\color{incolor}412}]:} \PY{c+c1}{\PYZsh{} create a matrix that represents the total effect of the operation}
          \PY{n}{gate\PYZus{}operation} \PY{o}{=} \PY{n}{q}\PY{o}{.}\PY{n}{gate\PYZus{}sequence\PYZus{}product}\PY{p}{(}\PY{n}{qc}\PY{o}{.}\PY{n}{propagators}\PY{p}{(}\PY{p}{)}\PY{p}{)}
          
          \PY{c+c1}{\PYZsh{} let it act on the state |01\PYZgt{}}
          \PY{n}{psi} \PY{o}{=} \PY{n}{q}\PY{o}{.}\PY{n}{ket}\PY{p}{(}\PY{l+s+s1}{\PYZsq{}}\PY{l+s+s1}{01}\PY{l+s+s1}{\PYZsq{}}\PY{p}{)}
          \PY{n}{psi\PYZus{}f} \PY{o}{=} \PY{n}{gate\PYZus{}operation} \PY{o}{*} \PY{n}{psi}
          
          \PY{c+c1}{\PYZsh{} investigate the gate action in the \PYZdq{}bit string ordered\PYZdq{} basis, we see}
          \PY{c+c1}{\PYZsh{} that we created the fully entangled state |+\PYZgt{}}
          \PY{n}{two\PYZus{}bit\PYZus{}basis\PYZus{}representation}\PY{p}{(}\PY{n}{psi\PYZus{}f}\PY{p}{)}
\end{Verbatim}


    \begin{Verbatim}[commandchars=\\\{\}]
|00> : 0.000+0.000j
|01> : 0.707+0.000j
|10> : 0.707+0.000j
|11> : 0.000+0.000j

    \end{Verbatim}

    In the lecture you have learned about a simple quantum error correction
scheme. It consists of a signal qubit (0) that should be uncorrupted and
two \emph{ancilla} qubits (1,2). By measuring the qubits 1 and 2 one can
determine, whether there has been a spin flip (\(\pi\)-rotation around
\(x\)) on any of the three qubits, without collapsing qubit 0. Therefore
one can correct the system for possible changes by applying a followup
gate.

\textbf{a)} \emph{(5 Points)} Implement the error correction scheme as a
\texttt{QubitCircuit} and verify the three predicted outcomes, depending
on which qubit undergoes a spin-flip, as well as that the quantum
information of the signal qubit is conserved. Visualize the gate
sequence (no penalty if it doesn't work due to software bugs).

    \begin{Verbatim}[commandchars=\\\{\}]
{\color{incolor}In [{\color{incolor}413}]:} \PY{n}{errors} \PY{o}{=} \PY{p}{[}\PY{k+kc}{None}\PY{p}{,} \PY{l+s+s1}{\PYZsq{}}\PY{l+s+s1}{RX}\PY{l+s+s1}{\PYZsq{}}\PY{p}{,} \PY{l+s+s1}{\PYZsq{}}\PY{l+s+s1}{RY}\PY{l+s+s1}{\PYZsq{}}\PY{p}{,} \PY{l+s+s1}{\PYZsq{}}\PY{l+s+s1}{RZ}\PY{l+s+s1}{\PYZsq{}}\PY{p}{]}
          \PY{n}{error\PYZus{}targets} \PY{o}{=} \PY{p}{[}\PY{l+m+mi}{0}\PY{p}{,} \PY{l+m+mi}{1}\PY{p}{,} \PY{l+m+mi}{2}\PY{p}{]}
          
          \PY{n}{one} \PY{o}{=} \PY{n}{q}\PY{o}{.}\PY{n}{basis}\PY{p}{(}\PY{l+m+mi}{2}\PY{p}{,}\PY{l+m+mi}{1}\PY{p}{)}
          \PY{n}{zero} \PY{o}{=} \PY{n}{q}\PY{o}{.}\PY{n}{basis}\PY{p}{(}\PY{l+m+mi}{2}\PY{p}{,}\PY{l+m+mi}{0}\PY{p}{)}
          \PY{n}{psi} \PY{o}{=} \PY{n}{q}\PY{o}{.}\PY{n}{Qobj}\PY{p}{(}\PY{p}{[}\PY{p}{[}\PY{l+m+mi}{1}\PY{p}{]}\PY{p}{,}\PY{p}{[}\PY{l+m+mi}{1}\PY{p}{]}\PY{p}{]}\PY{p}{)}\PY{o}{/}\PY{n}{np}\PY{o}{.}\PY{n}{sqrt}\PY{p}{(}\PY{l+m+mi}{2}\PY{p}{)}
          
          \PY{k}{for} \PY{n}{i} \PY{o+ow}{in} \PY{n}{error\PYZus{}targets}\PY{p}{:}
              \PY{n+nb}{print}\PY{p}{(}\PY{l+s+s1}{\PYZsq{}}\PY{l+s+s1}{Spin flip occurring on }\PY{l+s+si}{\PYZob{}\PYZcb{}}\PY{l+s+s1}{\PYZsq{}}\PY{o}{.}\PY{n}{format}\PY{p}{(}\PY{n}{i}\PY{p}{)}\PY{p}{)}
              \PY{n}{qc} \PY{o}{=} \PY{n}{q}\PY{o}{.}\PY{n}{QubitCircuit}\PY{p}{(}\PY{n}{N}\PY{o}{=}\PY{l+m+mi}{3}\PY{p}{)}
              
              \PY{c+c1}{\PYZsh{} this  is for plotting only}
              \PY{n}{qc}\PY{o}{.}\PY{n}{add\PYZus{}state}\PY{p}{(}\PY{l+s+s1}{\PYZsq{}}\PY{l+s+s1}{\PYZbs{}}\PY{l+s+s1}{Psi}\PY{l+s+s1}{\PYZsq{}}\PY{p}{,} \PY{p}{[}\PY{l+m+mi}{0}\PY{p}{]}\PY{p}{,} \PY{n}{state\PYZus{}type}\PY{o}{=}\PY{l+s+s1}{\PYZsq{}}\PY{l+s+s1}{input}\PY{l+s+s1}{\PYZsq{}}\PY{p}{)}
              \PY{n}{qc}\PY{o}{.}\PY{n}{add\PYZus{}state}\PY{p}{(}\PY{l+s+s1}{\PYZsq{}}\PY{l+s+s1}{0}\PY{l+s+s1}{\PYZsq{}}\PY{p}{,} \PY{p}{[}\PY{l+m+mi}{1}\PY{p}{]}\PY{p}{,} \PY{n}{state\PYZus{}type}\PY{o}{=}\PY{l+s+s1}{\PYZsq{}}\PY{l+s+s1}{input}\PY{l+s+s1}{\PYZsq{}}\PY{p}{)}
              \PY{n}{qc}\PY{o}{.}\PY{n}{add\PYZus{}state}\PY{p}{(}\PY{l+s+s1}{\PYZsq{}}\PY{l+s+s1}{0}\PY{l+s+s1}{\PYZsq{}}\PY{p}{,} \PY{p}{[}\PY{l+m+mi}{2}\PY{p}{]}\PY{p}{,} \PY{n}{state\PYZus{}type}\PY{o}{=}\PY{l+s+s1}{\PYZsq{}}\PY{l+s+s1}{input}\PY{l+s+s1}{\PYZsq{}}\PY{p}{)}
              
              \PY{c+c1}{\PYZsh{} initial entangled state preparation}
              \PY{n}{qc}\PY{o}{.}\PY{n}{add\PYZus{}gate}\PY{p}{(}\PY{l+s+s1}{\PYZsq{}}\PY{l+s+s1}{RX}\PY{l+s+s1}{\PYZsq{}}\PY{p}{,} \PY{n}{controls}\PY{o}{=}\PY{k+kc}{None}\PY{p}{,} \PY{n}{targets}\PY{o}{=}\PY{l+m+mi}{0}\PY{p}{,} \PY{n}{arg\PYZus{}value}\PY{o}{=}\PY{n}{pi}\PY{o}{/}\PY{l+m+mi}{2}\PY{p}{)}
              
              \PY{c+c1}{\PYZsh{} entangling psi with the two ancillae}
              \PY{n}{qc}\PY{o}{.}\PY{n}{add\PYZus{}gate}\PY{p}{(}\PY{l+s+s1}{\PYZsq{}}\PY{l+s+s1}{CNOT}\PY{l+s+s1}{\PYZsq{}}\PY{p}{,} \PY{n}{controls}\PY{o}{=}\PY{l+m+mi}{0}\PY{p}{,}\PY{n}{targets}\PY{o}{=}\PY{l+m+mi}{1}\PY{p}{)}
              \PY{n}{qc}\PY{o}{.}\PY{n}{add\PYZus{}gate}\PY{p}{(}\PY{l+s+s1}{\PYZsq{}}\PY{l+s+s1}{CNOT}\PY{l+s+s1}{\PYZsq{}}\PY{p}{,} \PY{n}{controls}\PY{o}{=}\PY{l+m+mi}{0}\PY{p}{,}\PY{n}{targets}\PY{o}{=}\PY{l+m+mi}{2}\PY{p}{)}
              
              \PY{c+c1}{\PYZsh{} apply spin flip to one of the qubits}
              \PY{n}{qc}\PY{o}{.}\PY{n}{add\PYZus{}gate}\PY{p}{(}\PY{l+s+s1}{\PYZsq{}}\PY{l+s+s1}{RX}\PY{l+s+s1}{\PYZsq{}}\PY{p}{,} \PY{n}{controls}\PY{o}{=}\PY{k+kc}{None}\PY{p}{,} \PY{n}{targets}\PY{o}{=}\PY{n}{i}\PY{p}{,} \PY{n}{arg\PYZus{}value}\PY{o}{=}\PY{n}{pi}\PY{p}{)}
             
              \PY{n}{qc}\PY{o}{.}\PY{n}{add\PYZus{}gate}\PY{p}{(}\PY{l+s+s1}{\PYZsq{}}\PY{l+s+s1}{CNOT}\PY{l+s+s1}{\PYZsq{}}\PY{p}{,} \PY{n}{controls}\PY{o}{=}\PY{l+m+mi}{0}\PY{p}{,} \PY{n}{targets}\PY{o}{=}\PY{l+m+mi}{2}\PY{p}{)}
              \PY{n}{qc}\PY{o}{.}\PY{n}{add\PYZus{}gate}\PY{p}{(}\PY{l+s+s1}{\PYZsq{}}\PY{l+s+s1}{CNOT}\PY{l+s+s1}{\PYZsq{}}\PY{p}{,} \PY{n}{controls}\PY{o}{=}\PY{l+m+mi}{0}\PY{p}{,} \PY{n}{targets}\PY{o}{=}\PY{l+m+mi}{1}\PY{p}{)}
              
              \PY{c+c1}{\PYZsh{} propagate a given input state through the error correction gate}
              \PY{n}{inputstate} \PY{o}{=} \PY{n}{q}\PY{o}{.}\PY{n}{ket}\PY{p}{(}\PY{l+s+s1}{\PYZsq{}}\PY{l+s+s1}{100}\PY{l+s+s1}{\PYZsq{}}\PY{p}{)}
              \PY{n}{operation} \PY{o}{=} \PY{n}{q}\PY{o}{.}\PY{n}{gate\PYZus{}sequence\PYZus{}product}\PY{p}{(}\PY{n}{qc}\PY{o}{.}\PY{n}{propagators}\PY{p}{(}\PY{p}{)}\PY{p}{)}
              
              \PY{n}{finalstate} \PY{o}{=} \PY{n}{operation}\PY{o}{*}\PY{n}{inputstate}
              
              \PY{c+c1}{\PYZsh{}print(operation)}
              \PY{c+c1}{\PYZsh{}print(inputstate)}
              \PY{n}{three\PYZus{}bit\PYZus{}basis\PYZus{}representation}\PY{p}{(}\PY{n}{finalstate}\PY{p}{)}
              \PY{n}{qc}\PY{o}{.}\PY{n}{png}
              \PY{n+nb}{print}\PY{p}{(}\PY{l+s+s1}{\PYZsq{}}\PY{l+s+se}{\PYZbs{}n}\PY{l+s+s1}{\PYZsq{}}\PY{p}{)}
\end{Verbatim}


    \begin{Verbatim}[commandchars=\\\{\}]
Spin flip occurring on 0
|000> : 0.000+0.000j
|001> : 0.000+0.000j
|010> : 0.000+0.000j
|011> : 0.000-0.707j
|100> : 0.000+0.000j
|101> : 0.000+0.000j
|110> : 0.000+0.000j
|111> : -0.707+0.000j


Spin flip occurring on 1
|000> : 0.000+0.000j
|001> : 0.000+0.000j
|010> : -0.707+0.000j
|011> : 0.000+0.000j
|100> : 0.000+0.000j
|101> : 0.000+0.000j
|110> : 0.000-0.707j
|111> : 0.000+0.000j


Spin flip occurring on 2
|000> : 0.000+0.000j
|001> : -0.707+0.000j
|010> : 0.000+0.000j
|011> : 0.000+0.000j
|100> : 0.000+0.000j
|101> : 0.000-0.707j
|110> : 0.000+0.000j
|111> : 0.000+0.000j



    \end{Verbatim}

    \begin{Verbatim}[commandchars=\\\{\}]
{\color{incolor}In [{\color{incolor}414}]:} \PY{n}{qc}\PY{o}{.}\PY{n}{png}
\end{Verbatim}

\texttt{\color{outcolor}Out[{\color{outcolor}414}]:}
    
    \begin{center}
    \adjustimage{max size={0.9\linewidth}{0.9\paperheight}}{output_11_0.png}
    \end{center}
    { \hspace*{\fill} \\}
    

    \section{Deutsch-Josza Algorithm}\label{deutsch-josza-algorithm}

    The Deutsch-Josza algorithm is an example for a (somewhat constructed)
problem in which a quantum computer outperforms a classical computer.
The problem is stated as follows: consider a Boolean function

\begin{align*}
f(\lbrace 0, 1 \rbrace^n) \longrightarrow \lbrace 0, 1 \rbrace
\end{align*}

mapping \(n\) (qu)bits of input to a single (qu)bit of output. We are
promised that the function \(f(x)\) is either a constant or a balanced
{[}1{]} function and want to find out which of both possibilities is
realized. Classically, to determine the nature of the function, we would
need to check more than half of the possible inputs, i.e.
\((2^n +1)/2\), to be sure. Using the Deutsch-Josza algorithm, the
computational effort consists of only a single evaluation of \(f(x)\)
and the measurement of \(n-1\) qubits, thus scaling linearly instead of
exponentially with \(n\).

{[}1{]} A balanced function maps the input to an output of 50\% each
\(0\) and \(1\).

    The implementation of the Deutsch-Josza algorithm is shown below. \(n\)
input qubits are initialized in the \(\ket{0}\) state, one target qubit
is initialized in the state \(\ket{1}\). The data qubits then subject to
a Hadamard transform \(H_n = H \otimes ... \otimes H\), the same is done
for the target qubit. The circuits kernel \(U_f\) ouputs the unvaried
data qubits and subjects \(\ket{y}\) to the operation
\(\ket{y \oplus f(x)}\). You might remember from the last paper that
\(\oplus\) equals a controled-NOT operation, here \(\ket{f(x)}\) is the
control and \(\ket{y}\) is the target qubit. After the kernel, the data
qubits again undergo a Hadamard transform and are then measured.

The nature of \(f(x)\) is determined in the following way: if the
function is constant, the measurement outcome will be \(\ket{0}^n\) with
probability \(1\), if it is balanced there will be \(0\) probability of
measuring the final state \(\ket{0}^n\)

    \begin{Verbatim}[commandchars=\\\{\}]
{\color{incolor}In [{\color{incolor}415}]:} \PY{n}{SVG}\PY{p}{(}\PY{l+s+s1}{\PYZsq{}}\PY{l+s+s1}{qc\PYZus{}intro\PYZus{}deutsch\PYZus{}josza.svg}\PY{l+s+s1}{\PYZsq{}}\PY{p}{)}
\end{Verbatim}

\texttt{\color{outcolor}Out[{\color{outcolor}415}]:}
    
    \begin{center}
    \adjustimage{max size={0.9\linewidth}{0.9\paperheight}}{output_15_0.pdf}
    \end{center}
    { \hspace*{\fill} \\}
    

    \begin{Verbatim}[commandchars=\\\{\}]
{\color{incolor}In [{\color{incolor}416}]:} \PY{n}{SVG}\PY{p}{(}\PY{l+s+s1}{\PYZsq{}}\PY{l+s+s1}{qc\PYZus{}intro\PYZus{}deutsch\PYZus{}josza1.svg}\PY{l+s+s1}{\PYZsq{}}\PY{p}{)}
\end{Verbatim}

\texttt{\color{outcolor}Out[{\color{outcolor}416}]:}
    
    \begin{center}
    \adjustimage{max size={0.9\linewidth}{0.9\paperheight}}{output_16_0.pdf}
    \end{center}
    { \hspace*{\fill} \\}
    

    \textbf{a)} \emph{(5 Points)} Below, you are given two examplary circuit
kernel for a constant/balanced \(f(x)\). Implement the Deutsch-Josza
algorithm on five qubits and verify the described outcome for both
kernels.

\emph{Implementation:} I recommend you to multiply single gate
operations from \texttt{qutip.gates} to your initial state vector
\texttt{qutip.ket(\textquotesingle{}10000\textquotesingle{})}. The
Hadamard gate is implemented in \texttt{qutip.gates.snot(N,\ target)},
the Hadamard transform is \texttt{qutip.hadamard\_transform(N)}.

    \begin{Verbatim}[commandchars=\\\{\}]
{\color{incolor}In [{\color{incolor}417}]:} \PY{k}{def} \PY{n+nf}{unary\PYZus{}constant}\PY{p}{(}\PY{n}{ket\PYZus{}input}\PY{p}{,} \PY{n}{N}\PY{o}{=}\PY{l+m+mi}{2}\PY{p}{)}\PY{p}{:}
              \PY{c+c1}{\PYZsh{} in the constant case we can choose f(x) = 0 or 1}
              \PY{n}{fx} \PY{o}{=} \PY{l+m+mi}{1}
              
              \PY{c+c1}{\PYZsh{} for 1 the operation |y + f(x)\PYZgt{} is equivalent to a NOT operation}
              \PY{c+c1}{\PYZsh{} for 0, the operation is equal to unity}
              \PY{k}{if} \PY{n}{fx} \PY{o}{==} \PY{l+m+mi}{1}\PY{p}{:}
                  \PY{n}{gate} \PY{o}{=} \PY{n}{q}\PY{o}{.}\PY{n}{gates}\PY{o}{.}\PY{n}{sqrtnot}\PY{p}{(}\PY{n}{N}\PY{o}{=}\PY{n}{N}\PY{p}{,} \PY{n}{target}\PY{o}{=}\PY{l+m+mi}{0}\PY{p}{)}\PY{o}{*}\PY{o}{*}\PY{l+m+mi}{2}
              \PY{k}{elif} \PY{n}{fx} \PY{o}{==}\PY{l+m+mi}{0}\PY{p}{:}
                  \PY{n}{gate} \PY{o}{=} \PY{n}{q}\PY{o}{.}\PY{n}{tensor}\PY{p}{(}\PY{p}{[}\PY{n}{q}\PY{o}{.}\PY{n}{qeye}\PY{p}{(}\PY{l+m+mi}{2}\PY{p}{)}\PY{p}{]}\PY{o}{*}\PY{n}{N}\PY{p}{)}
              
              \PY{n}{ket\PYZus{}output} \PY{o}{=} \PY{n}{gate} \PY{o}{*} \PY{n}{ket\PYZus{}input}
              
              \PY{k}{return} \PY{n}{ket\PYZus{}output}
          
          \PY{k}{def} \PY{n+nf}{unary\PYZus{}balanced}\PY{p}{(}\PY{n}{ket\PYZus{}input}\PY{p}{,} \PY{n}{N}\PY{o}{=}\PY{l+m+mi}{2}\PY{p}{)}\PY{p}{:}
              \PY{c+c1}{\PYZsh{} as an exemplary unary balanced function we use one, whose}
              \PY{c+c1}{\PYZsh{} output is 0 for bit N == 0 and 1 for bit N == 1}
              \PY{c+c1}{\PYZsh{} this means, the operation is a CNOT with control N and target 0}
              \PY{n}{cnot} \PY{o}{=} \PY{n}{q}\PY{o}{.}\PY{n}{gates}\PY{o}{.}\PY{n}{cnot}\PY{p}{(}\PY{n}{N}\PY{o}{=}\PY{n}{N}\PY{p}{,} \PY{n}{control}\PY{o}{=}\PY{n}{N}\PY{o}{\PYZhy{}}\PY{l+m+mi}{1}\PY{p}{,} \PY{n}{target}\PY{o}{=}\PY{l+m+mi}{0}\PY{p}{)}
              \PY{n}{ket\PYZus{}output} \PY{o}{=} \PY{n}{cnot} \PY{o}{*} \PY{n}{ket\PYZus{}input}
              
              \PY{k}{return} \PY{n}{ket\PYZus{}output}
\end{Verbatim}


    \begin{Verbatim}[commandchars=\\\{\}]
{\color{incolor}In [{\color{incolor}418}]:} \PY{c+c1}{\PYZsh{} the deutsch josza algorithm for N qubits, qubit 0 is the target, most significant bit first!}
          \PY{n}{N} \PY{o}{=} \PY{l+m+mi}{5}
          \PY{n}{psi\PYZus{}init} \PY{o}{=} \PY{n}{q}\PY{o}{.}\PY{n}{ket}\PY{p}{(}\PY{l+s+s1}{\PYZsq{}}\PY{l+s+s1}{1}\PY{l+s+s1}{\PYZsq{}}\PY{o}{+}\PY{l+s+s1}{\PYZsq{}}\PY{l+s+s1}{0}\PY{l+s+s1}{\PYZsq{}}\PY{o}{*}\PY{p}{(}\PY{n}{N}\PY{o}{\PYZhy{}}\PY{l+m+mi}{1}\PY{p}{)}\PY{p}{)}
          
          \PY{n}{psiH} \PY{o}{=} \PY{n}{q}\PY{o}{.}\PY{n}{tensor}\PY{p}{(}\PY{n}{q}\PY{o}{.}\PY{n}{snot}\PY{p}{(}\PY{n}{N}\PY{o}{=}\PY{l+m+mi}{1}\PY{p}{,} \PY{n}{target}\PY{o}{=}\PY{l+m+mi}{0}\PY{p}{)}\PY{p}{,} \PY{n}{q}\PY{o}{.}\PY{n}{hadamard\PYZus{}transform}\PY{p}{(}\PY{n}{N}\PY{o}{=}\PY{n}{N}\PY{o}{\PYZhy{}}\PY{l+m+mi}{1}\PY{p}{)}\PY{p}{)}\PY{o}{*} \PY{n}{psi\PYZus{}init}
          
          \PY{n}{psiU} \PY{o}{=} \PY{n}{unary\PYZus{}constant}\PY{p}{(}\PY{n}{psiH}\PY{p}{,} \PY{n}{N}\PY{o}{=}\PY{n}{N}\PY{p}{)}
          
          \PY{n}{psif} \PY{o}{=} \PY{n}{q}\PY{o}{.}\PY{n}{tensor}\PY{p}{(}\PY{n}{q}\PY{o}{.}\PY{n}{qeye}\PY{p}{(}\PY{l+m+mi}{2}\PY{p}{)}\PY{p}{,} \PY{n}{q}\PY{o}{.}\PY{n}{hadamard\PYZus{}transform}\PY{p}{(}\PY{n}{N}\PY{o}{=}\PY{n}{N}\PY{o}{\PYZhy{}}\PY{l+m+mi}{1}\PY{p}{)}\PY{p}{)} \PY{o}{*} \PY{n}{psiU}
          
          \PY{n}{five\PYZus{}bit\PYZus{}basis\PYZus{}representation}\PY{p}{(}\PY{n}{psif}\PY{p}{)}
\end{Verbatim}


    \begin{Verbatim}[commandchars=\\\{\}]
|00000> : -0.707+0.000j
|00001> : 0.000+0.000j
|00010> : 0.000+0.000j
|00011> : 0.000+0.000j
|00100> : 0.000+0.000j
|00101> : 0.000+0.000j
|00110> : 0.000+0.000j
|00111> : 0.000+0.000j
|01000> : 0.000+0.000j
|01001> : 0.000+0.000j
|01010> : 0.000+0.000j
|01011> : 0.000+0.000j
|01100> : 0.000+0.000j
|01101> : 0.000+0.000j
|01110> : 0.000+0.000j
|01111> : 0.000+0.000j
|10000> : 0.707+0.000j
|10001> : 0.000+0.000j
|10010> : 0.000+0.000j
|10011> : 0.000+0.000j
|10100> : 0.000+0.000j
|10101> : 0.000+0.000j
|10110> : 0.000+0.000j
|10111> : 0.000+0.000j
|11000> : 0.000+0.000j
|11001> : 0.000+0.000j
|11010> : 0.000+0.000j
|11011> : 0.000+0.000j
|11100> : 0.000+0.000j
|11101> : 0.000+0.000j
|11110> : 0.000+0.000j
|11111> : 0.000+0.000j

    \end{Verbatim}

    \begin{Verbatim}[commandchars=\\\{\}]
{\color{incolor}In [{\color{incolor}419}]:} \PY{c+c1}{\PYZsh{} the deutsch josza algorithm for N qubits, qubit 0 is the target, most significant bit first!}
          \PY{n}{N} \PY{o}{=} \PY{l+m+mi}{5}
          \PY{n}{psi\PYZus{}init} \PY{o}{=} \PY{n}{q}\PY{o}{.}\PY{n}{ket}\PY{p}{(}\PY{l+s+s1}{\PYZsq{}}\PY{l+s+s1}{1}\PY{l+s+s1}{\PYZsq{}}\PY{o}{+}\PY{l+s+s1}{\PYZsq{}}\PY{l+s+s1}{0}\PY{l+s+s1}{\PYZsq{}}\PY{o}{*}\PY{p}{(}\PY{n}{N}\PY{o}{\PYZhy{}}\PY{l+m+mi}{1}\PY{p}{)}\PY{p}{)}
          
          \PY{n}{psiH} \PY{o}{=} \PY{n}{q}\PY{o}{.}\PY{n}{tensor}\PY{p}{(}\PY{n}{q}\PY{o}{.}\PY{n}{snot}\PY{p}{(}\PY{n}{N}\PY{o}{=}\PY{l+m+mi}{1}\PY{p}{,} \PY{n}{target}\PY{o}{=}\PY{l+m+mi}{0}\PY{p}{)}\PY{p}{,} \PY{n}{q}\PY{o}{.}\PY{n}{hadamard\PYZus{}transform}\PY{p}{(}\PY{n}{N}\PY{o}{=}\PY{n}{N}\PY{o}{\PYZhy{}}\PY{l+m+mi}{1}\PY{p}{)}\PY{p}{)}\PY{o}{*} \PY{n}{psi\PYZus{}init}
          
          \PY{n}{psiU} \PY{o}{=} \PY{n}{unary\PYZus{}balanced}\PY{p}{(}\PY{n}{psiH}\PY{p}{,} \PY{n}{N}\PY{o}{=}\PY{n}{N}\PY{p}{)}
          
          \PY{n}{psif} \PY{o}{=} \PY{n}{q}\PY{o}{.}\PY{n}{tensor}\PY{p}{(}\PY{n}{q}\PY{o}{.}\PY{n}{qeye}\PY{p}{(}\PY{l+m+mi}{2}\PY{p}{)}\PY{p}{,} \PY{n}{q}\PY{o}{.}\PY{n}{hadamard\PYZus{}transform}\PY{p}{(}\PY{n}{N}\PY{o}{=}\PY{n}{N}\PY{o}{\PYZhy{}}\PY{l+m+mi}{1}\PY{p}{)}\PY{p}{)} \PY{o}{*} \PY{n}{psiU}
          
          \PY{n}{five\PYZus{}bit\PYZus{}basis\PYZus{}representation}\PY{p}{(}\PY{n}{psif}\PY{p}{)}
\end{Verbatim}


    \begin{Verbatim}[commandchars=\\\{\}]
|00000> : 0.000+0.000j
|00001> : 0.707+0.000j
|00010> : 0.000+0.000j
|00011> : 0.000+0.000j
|00100> : 0.000+0.000j
|00101> : 0.000+0.000j
|00110> : 0.000+0.000j
|00111> : 0.000+0.000j
|01000> : 0.000+0.000j
|01001> : 0.000+0.000j
|01010> : 0.000+0.000j
|01011> : 0.000+0.000j
|01100> : 0.000+0.000j
|01101> : 0.000+0.000j
|01110> : 0.000+0.000j
|01111> : 0.000+0.000j
|10000> : 0.000+0.000j
|10001> : -0.707+0.000j
|10010> : 0.000+0.000j
|10011> : 0.000+0.000j
|10100> : 0.000+0.000j
|10101> : 0.000+0.000j
|10110> : 0.000+0.000j
|10111> : 0.000+0.000j
|11000> : 0.000+0.000j
|11001> : 0.000+0.000j
|11010> : 0.000+0.000j
|11011> : 0.000+0.000j
|11100> : 0.000+0.000j
|11101> : 0.000+0.000j
|11110> : 0.000+0.000j
|11111> : 0.000+0.000j

    \end{Verbatim}

    \textbf{b)} \emph{(2 Points)} Describe where the speedup of the
Deutsch-Josza algorithm originates.


    % Add a bibliography block to the postdoc
    
    
    
    \end{document}
